\documentclass[english,letter,doublesided]{article}
\usepackage{rotating}
\newcommand{\G}{\overline{C_{2k-1}}}
\usepackage[latin9]{inputenc}
\usepackage{amsmath,calligra,mathrsfs,amsfonts}
\usepackage{amssymb}
\usepackage{lmodern}
\usepackage{mathtools}
\usepackage{enumitem}
\usepackage{pgf}
\usepackage{tikz}
\usepackage{tikz-cd}
\usepackage{relsize}

\usetikzlibrary{arrows, matrix}
%\usepackage{natbib}
%\bibliographystyle{plainnat}
%\setcitestyle{authoryear,open={(},close={)}}
\let\avec=\vec
\renewcommand\vec{\mathbf}
\renewcommand{\d}[1]{\ensuremath{\operatorname{d}\!{#1}}}
\newcommand{\pydx}[2]{\frac{\partial #1}{\newcommand\partial #2}}
\newcommand{\dydx}[2]{\frac{\d #1}{\d #2}}
\newcommand{\ddx}[1]{\frac{\d{}}{\d{#1}}}
\newcommand{\hk}{\hat{K}}
\newcommand{\hl}{\hat{\lambda}}
\newcommand{\ol}{\overline{\lambda}}
\newcommand{\om}{\overline{\mu}}
\newcommand{\all}{\text{all }}
\newcommand{\valph}{\vec{\alpha}}
\newcommand{\vbet}{\vec{\beta}}
\newcommand{\vT}{\vec{T}}
\newcommand{\vN}{\vec{N}}
\newcommand{\vB}{\vec{B}}
\newcommand{\vX}{\vec{X}}
\newcommand{\vx}{\vec {x}}
\newcommand{\vn}{\vec{n}}
\newcommand{\vxs}{\vec {x}^*}
\newcommand{\vV}{\vec{V}}
\newcommand{\vTa}{\vec{T}_\alpha}
\newcommand{\vNa}{\vec{N}_\alpha}
\newcommand{\vBa}{\vec{B}_\alpha}
\newcommand{\vTb}{\vec{T}_\beta}
\newcommand{\vNb}{\vec{N}_\beta}
\newcommand{\vBb}{\vec{B}_\beta}
\newcommand{\bvT}{\bar{\vT}}
\newcommand{\ka}{\kappa_\alpha}
\newcommand{\ta}{\tau_\alpha}
\newcommand{\kb}{\kappa_\beta}
\newcommand{\tb}{\tau_\beta}
\newcommand{\hth}{\hat{\theta}}
\newcommand{\evat}[3]{\left. #1\right|_{#2}^{#3}}
\newcommand{\prompt}[1]{\begin{prompt*}
		#1
\end{prompt*}}
\newcommand{\vy}{\vec{y}}
\DeclareMathOperator{\sech}{sech}
\DeclareMathOperator{\Spec}{Spec}
\DeclareMathOperator{\spec}{Spec}
\DeclareMathOperator{\spm}{Spm}
\DeclareMathOperator{\rad}{rad}
\newcommand{\mor}{\mathrm{Mor}}
\newcommand{\obj}{\mathrm{Obj}~}
\DeclarePairedDelimiter\abs{\lvert}{\rvert}%
\DeclarePairedDelimiter\norm{\lVert}{\rVert}%
\newcommand{\dis}[1]{\begin{align}
	#1
	\end{align}}
\renewcommand{\AA}{\mathbb{A}}
\newcommand{\LL}{\mathcal{L}}
\newcommand{\CC}{\mathbb{C}}
\newcommand{\DD}{\mathbb{D}}
\newcommand{\RR}{\mathbb{R}}
\newcommand{\NN}{\mathbb{N}}
\newcommand{\ZZ}{\mathbb{Z}}
\newcommand{\QQ}{\mathbb{Q}}
\newcommand{\Ss}{\mathcal{S}}
\newcommand{\OO}{\mathcal{O}}	
\newcommand{\BB}{\mathcal{B}}
\newcommand{\Pcal}{\mathcal{P}}
\newcommand{\FF}{\mathbb{F}}
\newcommand{\Ff}{\mathscr{F}}
\newcommand{\Gg}{\mathscr{G}}
\newcommand{\PP}{\mathbb{P}}
\newcommand{\Fcal}{\mathcal{F}}
\newcommand{\Gcal}{\mathcal{G}}
\newcommand{\fsc}{\mathscr{F}}
\newcommand{\afr}{\mathfrak{a}}
\newcommand{\bfr}{\mathfrak{b}}
\newcommand{\cfr}{\mathfrak{c}}
\newcommand{\dfr}{\mathfrak{d}}
\newcommand{\efr}{\mathfrak{e}}
\newcommand{\ffr}{\mathfrak{f}}
\newcommand{\gfr}{\mathfrak{g}}
\newcommand{\hfr}{\mathfrak{h}}
\newcommand{\ifr}{\mathfrak{i}}
\newcommand{\jfr}{\mathfrak{j}}
\newcommand{\kfr}{\mathfrak{k}}
\newcommand{\lfr}{\mathfrak{l}}
\newcommand{\mfr}{\mathfrak{m}}
\newcommand{\nfr}{\mathfrak{n}}
\newcommand{\ofr}{\mathfrak{o}}
\newcommand{\pfr}{\mathfrak{p}}
\newcommand{\qfr}{\mathfrak{q}}
\newcommand{\rfr}{\mathfrak{r}}
\newcommand{\sfr}{\mathfrak{s}}
\newcommand{\tfr}{\mathfrak{t}}
\newcommand{\ufr}{\mathfrak{u}}
\newcommand{\vfr}{\mathfrak{v}}
\newcommand{\wfr}{\mathfrak{w}}
\newcommand{\xfr}{\mathfrak{x}}
\newcommand{\yfr}{\mathfrak{y}}
\newcommand{\zfr}{\mathfrak{z}}
\newcommand{\Dcal}{\mathcal{D}}
\newcommand{\Ccal}{\mathcal{C}}
\newcommand{\Ical}{\mathcal{I}}
\usepackage{graphicx}
% Swap the definition of \abs* and \norm*, so that \abs
% and \norm resizes the size of the brackets, and the 
% starred version does not.
%\makeatletter
%\let\oldabs\abs
%\def\abs{\@ifstar{\oldabs}{\oldabs*}}
%
%\let\oldnorm\norm
%\def\norm{\@ifstar{\oldnorm}{\oldnorm*}}
%\makeatother
\newenvironment{subproof}[1][\proofname]{%
	\renewcommand{\qedsymbol}{$\blacksquare$}%
	\begin{proof}[#1]%
	}{%
	\end{proof}%
}

\usepackage{centernot}
\usepackage{dirtytalk}
\usepackage{calc}
\newcommand{\prob}[1]{\setcounter{section}{#1-1}\section{}}


\newcommand{\prt}[1]{\setcounter{subsection}{#1-1}\subsection{}}
\newcommand{\pprt}[1]{{\textit{{#1}.)}}\newline}
\renewcommand\thesubsection{\alph{subsection}}
\usepackage[sl,bf,compact]{titlesec}
\titlelabel{\thetitle.)\quad}
\DeclarePairedDelimiter\floor{\lfloor}{\rfloor}
\makeatletter	

\newcommand*\pFqskip{8mu}
\catcode`,\active
\newcommand*\pFq{\begingroup
	\catcode`\,\active
	\def ,{\mskip\pFqskip\relax}%
	\dopFq
}
\catcode`\,12
\def\dopFq#1#2#3#4#5{%
	{}_{#1}F_{#2}\biggl(\genfrac..{0pt}{}{#3}{#4}|#5\biggr
	)%
	\endgroup
}
\def\res{\mathop{Res}\limits}
% Symbols \wedge and \vee from mathabx
% \DeclareFontFamily{U}{matha}{\hyphenchar\font45}
% \DeclareFo\newcommand{\PP}{\mathbb{P}}ntShape{U}{matha}{m}{n}{
%       <5> <6> <7> <8> <9> <10> gen * matha
%       <10.95> matha10 <12> <14.4> <17.28> <20.74> <24.88> matha12
%       }{}
% \DeclareSymbolFont{matha}{U}{matha}{m}{n}
% \DeclareMathSymbol{\wedge}         {2}{matha}{"5E}
% \DeclareMathSymbol{\vee}           {2}{matha}{"5F}
% \makeatother

%\titlelabel{(\thesubsection)}
%\titlelabel{(\thesubsection)\quad}
\usepackage{listings}
\lstloadlanguages{[5.2]Mathematica}
\usepackage{babel}
\newcommand{\ffac}[2]{{(#1)}^{\underline{#2}}}
\usepackage{color}
\usepackage{amsthm}
\newtheorem{thm}{Theorem}[section]
\newtheorem*{thm*}{Theorem}
\newtheorem{conj}[thm]{Conjecture}
\newtheorem{cor}[thm]{Corollary}
\newtheorem{exle}[thm]{Example}
\newtheorem{lemma}[thm]{Lemma}
\newtheorem*{lemma*}{Lemma}
\newtheorem{problem}[thm]{Problem}
\newtheorem{prop}[thm]{Proposition}
\newtheorem*{prop*}{Proposition}
\newtheorem*{cor*}{Corollary}
\newtheorem{fact}[thm]{Fact}
\newtheorem*{prompt*}{Prompt}
\newtheorem*{claim*}{Claim}
\newcommand{\claim}[1]{\begin{claim*} #1\end{claim*}}
%organizing theorem environments by style--by the way, should we really have definitions (and notations I guess) in proposition style? it makes SO much of our text italicized, which is weird.
\theoremstyle{remark}
\newtheorem{remark}{Remark}[thm]
\newtheorem*{remark*}{Remark}

\theoremstyle{definition}
\newtheorem{defn}[thm]{Definition}
\newtheorem*{defn*}{Definition}
\newtheorem{notn}[thm]{Notation}
\newtheorem*{notn*}{Notation}
%FINAL
\newcommand{\course}{8254}
\newcommand{\due}{14 February 2018} 
\newcounter{hwn}
\setcounter{hwn}{2}
\RequirePackage{geometry}
\geometry{margin=.7in}
\usepackage{todonotes}
\title{MATH \course~ Homework \Roman{hwn}}
\author{David DeMark}
\date{\due}
\usepackage{fancyhdr}
\pagestyle{fancy}
\fancyhf{}
\rhead{David DeMark}
\chead{\due}
\lhead{MATH \course~ Homework \Roman{hwn}}
\cfoot{\thepage}
\renewcommand{\bar}{\overline}

% %%
%%
%%
%DRAFT

%\usepackage[left=1cm,right=4.5cm,top=2cm,bottom=1.5cm,marginparwidth=4cm]{geometry}
%\usepackage{todonotes}
% \title{MATH 8669 Homework 4-DRAFT}
% \usepackage{fancyhdr}
% \pagestyle{fancy}
% \fancyhf{}
% \rhead{David DeMark}
% \lhead{MATH 8669-Homework 4-DRAFT}
% \cfoot{\thepage}

%PROBLEM SPEFICIC
\renewcommand{\hom}{\mathrm{Hom}}
\newcommand{\lint}{\underline{\int}}
\newcommand{\uint}{\overline{\int}}
\newcommand{\hfi}{\hat{f}^{-1}}
\newcommand{\tfi}{\tilde{f}^{-1}}
\newcommand{\tsi}{\tilde{f}^{-1}}

\newcommand{\nin}{\centernot\in}
\newcommand{\seq}[1]{({#1}_n)_{n\geq 1}}
\newcommand{\Tt}{\mathcal{T}}
\newcommand{\card}{\mathrm{card}}
\newcommand{\setc}[2]{\{ #1\::\:#2 \}}
\newcommand{\idl}[1]{\langle #1 \rangle}
\newcommand{\cl}{\overline}
\newcommand{\id}{\mathrm{id} }
\newcommand{\im}{\mathrm{Im}}
\newcommand{\cat}[1]{{\mathrm{\bf{#1}}}}
%\usepackage[backend=biber,style=alphabetic]{biblatex}
%\addbibresource{algeo.bib}
\newcommand{\colim}{\varinjlim}
\newcommand{\clim}{\varprojlim}
\newcommand{\frp}{\mathop{\large {\mathlarger{\star}}}}
\newcommand{\restr}[2]{{\evat{#1}{#2}{}}}
\DeclareMathOperator{\codim}{codim}
\newcommand{\imp}[1]{\underline{#1}}
\newcommand{\ihm}{\imp{\hom}}
\newcommand{\him}{\ihm(\FF,\GG)}
\newcommand{\incla}{\hookrightarrow}
\newcommand{\pre}{\mathrm{pre}}
\newcommand{\Fp}{{\FF_P}}
\renewcommand{\thethm}{\arabic{section}.\Alph{thm}}
\newcommand{\gph}{\varphi}
\newcommand{\fv}[2]{\frac{x_{#1}}{x_{#2}}}
\newcommand{\va}{\vec{a}}
\newcommand{\vai}[1]{\va^{(#1)}}
\newcommand{\csch}{\cat{Sch}}
\newcommand{\cset}{\cat{Set}}
\newcommand{\aff}{\mathrm{aff}}
%\tikzcdset{column sep/tiny=.1cm}
\usepackage[backend=biber,style=alphabetic]{biblatex}
\addbibresource{algeo.bib}
\begin{document}\maketitle

\prob{1}
\begin{prompt*}
	We let $X=\spec \ZZ[x,y]/(y^2-2x)$, that is the curve in $\AA_\ZZ^1$ given by the equation $x=y^2/2$. Describe the fiber $X_p$ over each point $p\in \spec \ZZ$ as the spectrum of an algebra and enumerate the points $p$ over which $X_p\cong\AA^1_k$ for some field $k$. 
\end{prompt*}
\begin{proof}[Response]
We recall that the points of $\spec \ZZ$ are given by prime integers including the \say{prime at zero.\footnote{wait is that the prime at infinity?}} Then, the fiber of the map $X\to \spec \ZZ$ over $p\in \spec \ZZ$ is $X_p:=x_{(k(p))}:=X\times_{\spec \ZZ} \spec k(p)$. As $X$ is affine with ring of global sections $A=\ZZ[x,y]/(y^2-2x)$, $X_p\cong \spec \left(A\otimes_{\ZZ}k(p)\right)=\spec k(p)[x,y]/(y^2-2x)$. If $p=\idl{0}$, the generic point, then $k(p)=\ZZ_\idl{0}/\idl{0}\ZZ_{\idl{0}}=\QQ$, and for $p>0$, $k(p)=\ZZ_{\idl{p}}/\idl{p}\ZZ_{\idl{p}}=\left(\ZZ/p\right)_{\idl{p}}=\FF_p$. We note that in the case that $k(p)$ is not characteristic 2 (i.e. when $p\neq 2$), there is an isomorphism $k(p)[x,y]/\idl{y^2-2x}\to k(p)[t]$ by $y\mapsto t$, $x\mapsto \frac{t^2}{2}$ and hence $X_p\cong \AA^1_{k(p)}$. On the other hand, when $p=2$, $2x=0$ in $\FF_2$, so $A\otimes_\ZZ \FF_2=\FF_2[x,y]/\idl{y^2}$, which contains the nilpotent element $y$ and hence is not isomorphic to a polynomial ring over any field.
\end{proof}

\prob{2}
\begin{prop*}
For $X$ an $S$-scheme of finite type and $U\subset X$ an open subscheme, $U$ is an $S$-scheme of finite type.
\end{prop*}
For clarity's sake over the next two problems, we will give explicitly our definitions of finite type, taken from a survey of \cite[pp.84]{hsh}, \cite[pp.159]{FOAG} and in-class notes.
\begin{defn}~\label{defn:ft}
\begin{enumerate}[label=\textit{(\roman*)}]

	\item For $A$ a ring, $X$ is of finite type over $A$ if there is a \textit{finite} open cover $\{U_i\}$ of $X$ such that each $\OO_X(U_i)$ is an $A$-algebra of finite type in the ring-theoretic sense.
	\item $X$ is an $S$-scheme of finite type if for $\pi:X\to S$ there exists an open affine cover $\{V_i\}$ such that $\pi^{-1}(V_i)$ is of finite type over $V_i$. Equivalently, $X$ is of finite type if for any open affine $U\subset S$, $\pi^{-1}(U)$ is of finite type over $U$.
\end{enumerate}
\end{defn}
\begin{proof}[Proof of main proposition]
	We have that for any open affine $V\subset S$, $\pi^{-1}(V)$ is of finite type over $V$. We let $\{V_i\}$ be an open affine cover of $\pi(U)$ aaaaaaaaaaaaaaaaaaaand I just figured out that the proof I had written for this does not work oops!
\end{proof}


\prob{3}
\begin{prop*}
	If $X$ is an $S$-scheme of finite type by $\pi:X\to S$ and $S$ is a Noetherian scheme, then $X$ is Noetherian.
\end{prop*}
\begin{proof}
	As $S$ is Noetherian, there exists an affine open cover $\{U_i\}$ of $S$ where each $U_i\cong \spec A_i$ for a Noetherian ring $A_i$. As $S$ is necessarily quasi-compact, we may assume $\{U_i\}_{i=1}^m$ is a finite set without loss of generality. By Definition \ref{defn:ft}, $\pi^{-1}(U_i)$ is of finite type over $U_i$, that is there exists a finite open affine cover $\{V_{ji}\}_{j=1}^{n_i}$ of $\pi^{-1}(U_i)$ where $V_{ji}\cong\spec B_{ji}$ and $B_{ij}$ is a finite $A_i$-algebra. By the Hilbert Basis Theorem, $B_{ij}$ is then Noetherian, and hence the affine scheme $V_{ji}$ is Noetherian. Thus, $X$ has a finite open cover $\{V_{ji}\}_{i,j=1}^{i=m,j=n_i}$ by Noetherian affine subschemes, and is therefore itself Noetherian.	
\end{proof}
%\prob{4}

%\prob{5}

\prob{6}
\prt{1}
\begin{prop*}
We let $R$ be a ring of dimension 0. Then, $R=Q(R)$. 
\end{prop*}
\begin{proof}
	We let $x\in R$ be a non-unit. Then, there is some $\pfr\ni x$ with $\pfr$ necessarily minimal. Thus, $\frac{x}{1}\in \pfr R_\pfr\triangleleft R_\pfr$. As $\pfr R_\pfr$ is the only prime ideal of $R_\pfr$, we have that $\frac{x}{1}$ is nilpotent in $R_\pfr$. We let $k\geq 1$ be such that $(\frac{x}{1})^{k-1}\neq 0$ but $(\frac{x}{1})^k=0$. Then, there is some $u$ in $R\setminus \pfr$ such that $u(x^k-0)=0$ in $R$, i.e. $ux^k=0$. Thus, $x$ is a zero-divisor.
\end{proof}
\prt{2}
\begin{lemma}
	We let $R$ be a reduced ring, $Z$ the set of zero-divisors in $R$, and $\{\pfr_i\}_{i\in \Ical}$ the (possibly infinite) set of minimal prime ideals of $R$. Then, $Z\subset \bigcup_{i\in \Ical}\pfr_i$. \label{6blem}
\end{lemma}
\begin{subproof}[Proof of Lemma \ref{6blem}]
We let $x\in Z$. Then, there is some $y\in R$ such that $xy=0$. As $\bigcap_{i\in \Ical}\pfr_i=0$ in a reduced ring, we have that there is some $\pfr_i\centernot\ni y$. However, we have that $xy=0\in \pfr_i$. Thus, $x\in \pfr_i$. 
\end{subproof}
\begin{cor}[Main proposition for problem 6b]
	If $R$ is reduced, has only finitely many minimal prime ideals, and has $R=Q(R)$, then $\dim R\leq 0$. \label{6bcor}
\end{cor}
\begin{proof}[Proof of Corollary \ref{6bcor}]
	We let the minimal prime ideals of $R$ be the set $\{\pfr_1,\ldots,\pfr_n\}$, and let $I\triangleleft R$ be any proper ideal. Then, $I$ consists of zero-divisors as else it contains a unit, so $I\subset \bigcup_{i=1}^n\pfr_i$. By the theorem of prime avoidance (see for instance \cite[\S3.2]{eis}), we now have that $I\subset \pfr_i$ for some $1\leq i \leq n$. Thus, in $R$, any proper ideal is contained in some minimal prime, so the minimal primes are indeed maximal and thus $\dim R\leq 0$. 
\end{proof}
\prob{7}
\begin{prop*}
For $X$ a scheme, $x\in X$, $\codim \cl{\{x\}}=\dim \OO_{X,x}$.
\end{prop*}
We begin with a lemma
\begin{lemma}
	We let $X$ be a topological space and $\{U_i\}_{i\in \Ical}$ an open cover of $X$. Then, $\dim X=\sup_{i\in \Ical}\dim U_i$. Similarly, for any set $S\subset U_i$ for some $U_i$, we have that $\codim_XS=\codim_{U_i} S$ \label{7lemc}
\end{lemma}
\begin{subproof}[Proof of Lemma \ref{7lemc}]
	We note that, trivially, $\dim X\geq \sup_{i\in \Ical}\dim U_i$. We show that $\dim X\leq \sup_{i\in \Ical}\dim U_i$. We let $Z_0\subsetneq\ldots\subsetneq Z_r$ be any chain of closed irreducible subsets in $X$. Then, there is some $U_0$ such that $U_0\cap Z_0\neq \emptyset$. We claim that $U_0\cap Z_{i-1}\neq U_0\cap Z_i$ for any $1<i\leq r$. Indeed, suppose for the sake of contradiction $U_0\cap Z_{i-1}=U_0\cap Z_i$ for some $i$. Then, $Z_{i}\cap U_0^c$ is closed, proper, and nonempty as $\emptyset \neq Z_i\setminus Z_{i-1}\subset Z_i\cap U_0^c$. Furthermore, $Z_i\cap Z_{i-1}$ is closed and nonempty and proper as $Z_i\neq Z_{i-1}$. Then, $Z_i=\left(Z_i\cap U_0^c\right)\cup (Z_i\cap Z_{i-1})$, contradicting irreducibility of $Z_i$. Thus, for any $r\in \NN$ such that there exists a chain of length $r$ of closed irreducible subsets in $X$, there is some $U_i$ such that a chain of equal lenght exists in $U_i$, so indeed $\dim X\leq \sup_{i\in \Ical}\dim U_i$.
	
	The proof of the second statement of the lemma is exactly identical, noting that $U_i$ will play the role of $U_0$ in the above paragraph.
\end{subproof}
\begin{proof}[Proof of main proposition]
	We let $\{U_i\}_{i\in \Ical}$ be any affine open cover of $X$ with $\spec A\cong U_0\ni x$. Then, by the lemma, it suffices to show the proposition relative to the affine scheme $U_0$. Hence, without loss of generality, we let $X\cong \spec A$ be affine. Then, the codimension of $\cl{\{x\}}$ is the supremum of the lengths chains of primes in $A$ contained in $\pfr$, as those correspond to irreducible closed subsets containing $\cl{\{x\}}$ by the inclusion-reversing correspondence of the Nullstellensatz. Then, $\OO_{X,x}\cong A_\pfr$ where $x$ corresponds to the prime $\pfr\triangleleft A$. By a basic theorem of localization, there is an inclusion-preserving bijection between primes of $A_\pfr$ and those contained in $\pfr$ of $A$, and hence $\dim \OO_{X,x}$ is as well the supremum of the lengths chains of primes in $A$ contained in $\pfr$, thus completing our proof.
\end{proof}
\begin{cor*}
The dimension of a scheme is the supremum of the Krull dimension of its stalks
\end{cor*}
\begin{proof}
	We note that all chains of irreducible closed subspaces in $X$ must have a minimal element $Z_0$. Thus, the dimension of $X$ is the supremum of the codimensions of all irreducible closed subspaces. As those correspond to $V(\pfr)$ for some $\pfr\triangleleft \OO_{U}(U)$ for some affine open $U$, they are necessarily the closure of the point $\{x\}$ where $x$ corresponds to $\pfr$ in the isomorphism $U\cong \spec \OO_U(U)$.
\end{proof}
\printbibliography
\end{document}